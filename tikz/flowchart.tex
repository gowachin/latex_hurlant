% \documentclass[tikz]{standalone}
% \usepackage[utf8]{inputenc}

% \usetikzlibrary{shapes.geometric, arrows}

% \tikzstyle{startstop} = [rectangle, rounded corners, minimum width=3cm, minimum height=1cm,text centered, draw=black, fill=red!30]
% \tikzstyle{io} = [trapezium, trapezium left angle=70, trapezium right angle=110, minimum width=3cm, minimum height=1cm, text centered, draw=black, fill=blue!30]
% \tikzstyle{process} = [rectangle, minimum width=3cm, minimum height=1cm, text centered, text width=3cm, draw=black, fill=orange!30]
% \tikzstyle{decision} = [diamond, minimum width=3cm, minimum height=1cm, text centered, draw=black, fill=green!30]
% \tikzstyle{arrow} = [thick,->,>=stealth]

% \begin{document}

% \begin{tikzpicture}[node distance=2cm]

% \node (start) [startstop] {};
% \node (in1) [io, below of=start] {Input};
% \node (pro1) [process, below of=in1] {Process 1};
% \node (dec1) [decision, below of=pro1, yshift=-0.5cm] {Decision 1};
% \node (pro2a) [process, below of=dec1, yshift=-0.5cm] {Process 2a text text text text text text text text text text};
% \node (pro2b) [process, right of=dec1, xshift=2cm] {Process 2b};
% \node (out1) [io, below of=pro2a] {Output};
% \node (stop) [startstop, below of=out1] {Stop};



% \draw [arrow] (start) -- (in1);
% \draw [arrow] (in1) -- (pro1);
% \draw [arrow] (pro1) -- (dec1);
% \draw [arrow] (dec1) -- node[anchor=east] {yes} (pro2a);
% \draw [arrow] (dec1) -- node[anchor=south] {no} (pro2b);
% \draw [arrow] (pro2b) |- (pro1);
% \draw [arrow] (pro2a) -- (out1);
% \draw [arrow] (out1) -- (stop);


% \end{tikzpicture}

% \end{document}
\documentclass[tikz,border=3mm]{standalone}

\usetikzlibrary{chains}
\usetikzlibrary{shapes}

\definecolor{strange}{RGB}{230,244,253}
\begin{document}
\begin{tikzpicture}[block/.style={draw,fill=strange,minimum height=2.5em},
    font=\sffamily,>=stealth]
 \begin{scope}[start chain=A going below,node distance=1em,
    local bounding box=buffers]
  \foreach \X in {1,...,4}
  {\node[block,on chain,minimum width=8em]{Buffer};
  \draw[<-,thick] (A-\X.west) -- node[above]{$m_\X(t)$} ++ (-4em,0) ;
  }
 \end{scope} 
 \node[right=6em of buffers,align=center,inner sep=0.5pt,fill=strange,
    circle,draw] (SO) {Scan\\operations};
 \foreach \X in {1,...,4} 
  {
      \draw[->,thick] (A-\X.east) -- node[above]{$m_\X(t)$} ++ (4em,0) -- (SO);
  }
 \node[right=4em of SO,block,minimum width=6em](M){Modem};
 \draw[->,thick](SO) -- node[above](mc){$m_c(t)$} (M);
 \draw[->,thick](M.east) -- node[above](s){$s(t)$} ++(4em,0);
 \draw[<-] ([yshift=-1ex]mc.south) -- ++ (-1em,-3em)
  node[below,align=center]{Bla\\ bla};
 \draw[<-] ([yshift=-1ex]s.south) -- ++ (1em,-3em)
  node[below,align=center]{Blub\\ blub};
\end{tikzpicture}


\begin{tikzpicture}[block/.style={draw,fill=strange,minimum height=2.5em},
    font=\sffamily,>=stealth]

    \begin{scope}[%start chain=A going below,node distance=1em,
                  local bounding box=buffers]
        \node[block,minimum width=6em](remote){Remote github};
        \node[right= 8em of remote, block,minimum width=6em](local){Fichiers};
    \end{scope} 

    \node[below= 6em of buffers,align=center, block,minimum width=6em](repo){Repository \\Git local};

    %\draw[->, thick](remote) edge[bend left] node[midway, fill=white]{Pull} (local);
    \draw[->, thick](remote) edge[bend left] node[midway, fill=white]{Pull} (repo);
    %\draw[->, thick](remote) edge[dashed] node[midway, fill=white]{Fetch} (repo);
    \draw[->, thick](remote) edge node[midway, fill=white]{Clone} (local);
    \draw[->,thick](local) edge[bend left] node[midway, fill=white]{Add} (repo);
    \draw[->,thick](repo) edge[bend left] node[midway, fill=white]{Push} (remote);
    \draw[->,thick](repo) edge[loop right] node[midway, fill=white]{Commit} (repo);
    

\end{tikzpicture}


\end{document}